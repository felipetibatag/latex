\documentclass{elegantbook}
\usepackage{soul}
\usepackage{tikzsymbols}
\usepackage{hyperref}
\usepackage{multirow}
\begin{document}
\tableofcontents
\mainmatter
\hypersetup{pageanchor=true}
\chapter{Basic One}
    \section{Numbers}
    \begin{center}
        \begin{tabular}{c c c}
            \hline
            \# & Cardinal & Ordinal\\
            \hline
            1&One&First\\
            2&Two&Second\\
            3&Three&Third\\
            4&Four&Fourth\\
            5&Five&Fifth\\
            6&Six&Sixth\\
            7&Seven&Seventh\\
            8&Eight&Eighth\\
            9&Nine&Ninth\\
            10&Ten&Tenth\\
            11&Eleven&Eleventh\\
            12&Twelve&Twelfth\\
            13&Thirteen&Thirteenth\\
            14&Fourteen&Fourteenth\\
            15&Fifteen&Fifteenth\\
            16&Sixteen&Sixteenth\\
            17&Seventeen&Seventeenth\\
            18&Eighteen&Eightteenth\\
            19&Nineteen&Nineteenth\\
            20&Twenty&Twentieth\\
            21&Tweny-one&Twenty-First\\
            30&Thirty&Thirtieth\\
            40&Forty&Fortieth\\
            50&Fifty&Fiftieth\\
            60&Sixty&Sixtieth\\
            70&Seventy&Seventieth\\
            80&Eigthy&Eightieth\\
            90&Ninety&Ninetieth\\
            100&One hundred&Hundredth\\
            500&Five hundred&Five Hundredth\\
            1.000&One thousand&Thousandth\\
            1.500&One thousand five hundred&One Thousandth five Hundredth\\
            100.000&One hundred thousand&hundred Thousandth\\
            1000.000&One million&Millionth\\
        \end{tabular}
    \end{center}
    \begin{note}
        \textbf{Million} is ok but  never \st{Millions}
    \end{note}
    \section{WH Questions}
    \begin{corollary}{Syntax WH Questions}{}
        WH + V+ \dSmiley[2]+ C
    \end{corollary}

    \begin{tabular}{|c|c|}
        \hline
        What&Normally used for talking about \textbf{things}\\
        \hline
        When&Normally used for talking about \textbf{time}\\
        \hline
        Where&Normally used for talking about \textbf{places}\\
        \hline
        Why&Normally used for talking about \textbf{rasons}\\
        \hline
        How&Normally used for talking about \textbf{manner}\\
        \hline
        Which&Normally used for talking about \textbf{Options}\\
        \hline
        Who&Normally used for talking about \textbf{people}\\
        \hline
    \end{tabular}
    \\
    \begin{example}
        \begin{enumerate}
            \item What is our favorite car's brand?
            \item Which is your favorite color?
            \item Where are you?
            \item Who is you mother?
            \item What is your name mother?
            \item How are you today?
            \item Is he your brother?
            \item Are they teachers?
            \item Is it a nice car?
            \item Are we ready for the job?
            \item Is she a good person?
        \end{enumerate}
    \end{example}
    \section{Countries and Nationalities}
    \begin{tabular}{|c|c|}
        \hline
        \textbf{Country}&\textbf{Nationality}\\
        \hline
        Egypt&Egyptian\\
        \hline
        Brazil&Brazilian\\
        \hline
        India&Indian\\
        \hline
        Australia&Australian\\
        \hline
        Colombia&Colombian\\
        \hline
        Canada&Canadian\\
        \hline
        Korea&Korean\\
        \hline
        Malasya&Malasyan\\
        \hline
        Scotland&Scotish\\
        \hline
        Ireland&Irish\\
        \hline
        Vietnam&Vietnamese\\
        \hline
        Greece&Greek\\
        \hline
        Germany&German\\
        \hline
    \end{tabular}
    \begin{note}
        To see more countries in \url{http://www.esldesk.com/vocabulary/countries}
    \end{note}
    \section{Regular Plural Forms}
    \subsection{Nouns Ending In Sibilants}
    \begin{property}
        Sounds that sound as S as \emph{ch,sh,x,z,s,o}
    \end{property}
    \begin{property}
        If the noun end in \emph{-che} add \emph{-es}
    \end{property}
    \begin{property}
        If the noun end in \emph{-e} add \emph{-s}
    \end{property}
    \begin{center}
        \begin{tabular}{c|c}
            \textbf{Singular}&\textbf{Plural}\\
            \hline
            One bo\underline{x}&two box\underline{es}\\
            a Swuitcas\underline{e}&Two suitcase\underline{s}\\
            a Sandwi\underline{ch}&Two Sandwich\underline{es}\\
            g\underline{o}&go\underline{es}\\
        \end{tabular}
    \end{center}
    \subsection{Nouns Ending In \emph{y}}
    \subsubsection{\emph{y} After a \emph{\underline{Consonant}}}
    \begin{property}
        Change \emph{y} to \emph{i} and then add \emph{es}
    \end{property}
    \begin{center}
        \begin{tabular}{c|c}
            \textbf{Singular}&\textbf{Plural}\\
            \hline
            a cit\underline{y}&Two cit\underline{ies}\\
            a lad\underline{y}&Two lad\underline{ies}\\
        \end{tabular}
    \end{center}
    \subsubsection{\emph{y} After a \emph{\underline{Vowel}}}
    \begin{property}
        Add \emph{s}
    \end{property}
    \begin{center}
        \begin{tabular}{c|c}
            \textbf{Singular}&\textbf{Plural}\\
            \hline
            a bo\underline{y}&Two boy\underline{s}\\
            a da\underline{y}&Two day\underline{s}\\
        \end{tabular}
    \end{center}
    \subsection{Nouns Ending in \emph{f} or \emph{fe}}
    \begin{property}
        Add \emph{s}.
    \end{property}
    \begin{property}
        Add \emph{s} for words ending in \emph{ff}.
    \end{property}
    \begin{center}
        \begin{tabular}{c|c}
            \textbf{Singular}&\textbf{Plural}\\
            \hline
            a roo\underline{f}&Two roof\underline{s}\\
            a cli\underline{ff}&Two cliff\underline{s}\\
            a sheri\underline{ff}&Two sheriff\underline{s}\\
        \end{tabular}
    \end{center}
    \subsection{Substitute with \emph{ves}}
    \begin{center}
        \begin{tabular}{c|c}
            \textbf{Singular}&\textbf{Plural}\\
            \hline
            cal\underline{f}&cal\underline{ves}\\
            hal\underline{f}&hal\underline{ves}\\
            li\underline{fe}&li\underline{ves}\\
            wi\underline{fe}&wi\underline{ves}\\
            yoursel\underline{f}&yoursel\underline{ves}
        \end{tabular}
    \end{center}
    \subsection{Nound ending in \emph{o}}
    \begin{note}
        To this there isn't a rule when add \textbf{s} or \textbf{es}.
    \end{note}
    \begin{property}
        Add \emph{-s} for this words:
    \end{property}
    \begin{center}
        \begin{tabular}{c|c}
            \textbf{Singular}&\textbf{Plural}\\
            \hline
            A disco&Two disco\underline{s}\\
            A piano&Two piano\underline{s}\\
            A photo&Two photo\underline{s}\\
        \end{tabular}
    \end{center}
    \begin{property}
        Add \emph{-es} for this words:
    \end{property}
    \begin{center}
        \begin{tabular}{c|c}
            \textbf{Singular}&\textbf{Plural}\\
            \hline
            A tomato&Two tomato\underline{es}\\
            A potato&Two potato\underline{es}\\
            A hero&Two hero\underline{es}\\
        \end{tabular}
    \end{center}
    \section{Irregular Plural Forms}
    \begin{center}
        \begin{tabular}{c|c}
            \textbf{Singular}&\textbf{Plural}\\
            \hline
            a man&two men\\
            a woman&two women\\
            a child&two children\\
            a mouse&two mice\\
            a tooth&two teeth\\
            a goose&two geese\\
            a foot&two feet\\
            a ox&two oxen\\
        \end{tabular}
    \end{center}
    \section{Don't Change}
    \begin{center}
        \begin{tabular}{l|l}
            \textbf{Singular}&\textbf{Plural}\\
            \hline
            1 sheep&2 sheep\\
            1 deer&2 deer\\
            1 fish&2 fish\\
            1 series&2 series\\
            1 species&2 species\\
        \end{tabular}
    \end{center}

    \begin{note}
        \textbf{a} and \textbf{an} ONLY use for SINGULAR:
        \begin{itemize}
            \item \textbf{a} car
            \item \textbf{an} apple
            \item \textbf{an} orange
        \end{itemize}
    \end{note}

    \section{Demostrative Pronouns}
    \begin{center}
        \begin{tabular}{l|l}
            \textbf{Demostrative Pronouns}&\textbf{Significado}\\
            \hline
            This&esto/este/esta\\
            That&eso/esa/ese - aquél/aqello/aquella\\
            These&estos/estas\\
            Those&esos/esas - aquellos/aquellas
        \end{tabular}
        \begin{tabular}{c|c|c}
            &\textbf{Near}&\textbf{Far}\\
            \hline
            \textbf{Singular}&\textbf{This} is an apple&\textbf{That} is an apple\\
            \textbf{Plural}&\textbf{These} are apples&\textbf{Those} are apples\\
        \end{tabular}
    \end{center}
    \section{Possessive Pronouns And Adjectives}
        \begin{tabular}{|c|c|c|c|c|c}
            \hline
            \textbf{Subject}&\textcolor{red}{\textbf{Possessive Adjective}}&\textbf{Significado}&\textcolor{red}{\textbf{Possessive Pronoun}}&\textbf{Significado}\\
            \hline
            I&My&Mi(s)&Mine&Mio(s)\\
            You&Your&Mi(s)&Yours&tuyo/suyo\\
            He&His&sus/de él&His&Suyo(s)\\
            She&Her&sus/de ella&Hers&Suyo(s)\\
            It&Its&su(s)&Its&Su\\
            We&Our&nuestro/de nosotros&Ours&nuestro\\
            They&Their&sus/de ellos&Theirs&suyo(s)\\
            \hline
        \end{tabular}
        \begin{corollary}{}{}
            Possessive Adjective: \textbf{Después} de uno de ellos va sí o sí un sustantivo\\
            Possessive Pronoun: El sustantivo va \textbf{antes}, después de alguno de ellos \textbf{no} va nada.
        \end{corollary}
        \begin{example}\\
            \textsl{Possessive Adjective}
            \begin{itemize}
                \item This is \textbf{my} car.
                \item This is \textbf{your} book.
            \end{itemize}
            \textsl{Possessive Pronoun}
            \begin{itemize}
                \item This car is \textbf{mine}.
                \item This is house is \textbf{yours}.
            \end{itemize}
        \end{example}
        \section{Reflexive Pronouns}
        \begin{tabular}{|l|l|}
            \hline
            myself&yo mismo, a mi\\
            himself&él mismo, a si mismo\\
            herself&ella misma, a si misma\\
            itself&el mismo, a si mismo\\
            yourselves&ustedes mismos\\
            yourself&tú mismo, usted mismo\\
            ourselves&nosotros mismos\\
            themselves&ellos mismos\\
            \hline
        \end{tabular}
        \section{Present Simple Do-Does}
        \begin{corollary}{}{}
            WH + Do/Does + \dSmiley[2] + V + C
            
        \end{corollary}
        Describing habits or rutines.\\
        \begin{tabular}{|c|c|c|}
            \hline
            &\textbf{Positive}&\textbf{Negative}\\
            \hline
            I&\multirow{4}{10em}{Listen to the music on the bus}&\multirow{4}{10em}{Don't listen to music}\\
            We\\
            You\\
            They\\
            \hline
            He&\multirow{3}{10em}{Listen\underline{s} to the music on the bus}&\multirow{3}{10em}{Doesn't listen to music}\\
            She\\
            It\\
            \hline
        \end{tabular}

        \begin{tabular}{|c|c|c|}
            \hline
            \multicolumn{3}{|c|}{Interrogative}\\
            \hline
            &\textbf{Positive Answer}&\textbf{Negative Answer}\\
            \hline
            Do we listen to music?&Yes,We do&No,We don't\\
            \hline
            Does he listen to music?&Yes, He does&No, He doesn't\\
            \hline
        \end{tabular}
        \\
        \begin{example}
            \begin{enumerate}
                \item Do you like your English classes?
                \item Do they go running every day?
                \item Do you chat with friends a lot?
                \item Do you like eating junk food?
                \item Do they watch football on TV?
                \item Do you go to the cinema a lot?
            \end{enumerate}
        \end{example}

        \begin{example}
            \begin{itemize}
                \item My brother works on monday
                \item [?] Does my brother work on monday ?
                \item [+]yes, he does
                \item [-]no, he doesn't
                \item [-]no, he does not
                \item The store opens early
                \item [?] Does the store open early ?
                \item [+]yes, it does
                \item [-]no, it doesn't
                \item [-]no, it does not
                \item [?]What does time it start ?
                \item [?]Where does it leave from ?
                \item [?]When does the tour finish ?
                \item [?]How much does it ?
                \item [?]Do you take credit cards ?
            \end{itemize}
        \end{example}
        \section{Daily Routines}
        \begin{center}
            \begin{tabular}{|l|l|}
                \hline
                Get up&Levantarse\\
                Get home&Llegar a casa\\
                Leave home&Salir de casa\\
                Go to bed&Ir a la cama\\
                Have lunch&Almorzar\\
                Finish work&Terminar la jornada laboral\\
                Start work&Empezar a trabajar\\
                Have dinner&Comer/cenar\\
                Have breakfast&Desayunar\\
                \hline
            \end{tabular}
        \end{center}
        \section{What Time Is It?}
        \textcolor{red}{\textbf{insertart imagen de horas, reloj}}\\
        \begin{tabular}{cl}
            8:15&Quarter past eight\\
            9:20&Twenty past nine\\
            10:35&Twenty-five to eleven\\
            5:55&Five to six\\
            4:30&Half past four\\
        \end{tabular}
        \section{Making Request}
        \begin{itemize}
            \item Can I have a sandwich please?
            \item Could I have a single to Sidney please? \emph{Más formal}
        \end{itemize}
        \section{Family}
        \textcolor{red}{\textbf{insertar imagen de parentesco, familia}}
        \section{Have/Has Got}
        \begin{corollary}{Syntax}{}
            Have + \dSmiley[2] + got + c\\
            Has + \dSmiley[2] + got + c
            \textbf{\\I have blue eyes == I have got blue eyes}
        \end{corollary}
        \begin{tabular}{|c|l|l|}
            \hline
            &\textbf{Positive}&\textbf{Negative}\\
            \hline
            I&\multirow{4}{10em}{Have got a phone}&They haven't got any sisters\\
            We&&He hasn't got a car\\
            You&&He hasn't any car\\
            They\\
            \hline
        \end{tabular}

        \vspace{1em}
        \begin{tabular}{|c|c|c|}
            \hline
            \multicolumn{3}{|c|}{Interrogative}\\
            \hline
            &Positive&Negative\\
            \hline
            Have we got any cousin?&yes, we have&no, we haven't\\
            \hline
        \end{tabular}
        \\
        \begin{example}
            \\\emph{Negative}
            \begin{itemize}
                \item [+]I \textcolor{red}{have} blue eyes
                \item [+]I \textcolor{red}{have got} blue eyes
                \item [?] \textcolor{red}{\textbf{Do}} I \textcolor{red}{\textbf{have}} blue eyes?
                \item [?] \textcolor{red}{\textbf{Have}} I \textcolor{red}{\textbf{got}} blue eyes?
                \item [-]I \textcolor{red}{\textbf{don't have}} blue eyes.
                \item [-]I \textcolor{red}{\textbf{haven't got}} a blue eyes.
                \item [+]You have a new car
                \item [-]You \textcolor{red}{\textbf{don't have}} a new car.
                \item [-]You \textcolor{red}{\textbf{haven't got}} a new car.
                \item [+]He \textcolor{red}{\textbf{has}} got a headache
                \item [?]\textcolor{red}{\textbf{Does}} He \textcolor{red}{\textbf{have}} a headache?
                \item [?]\textcolor{red}{\textbf{Has}} he \textcolor{red}{\textbf{got}} a headache?
                \item [-]He \textcolor{red}{\textbf{doesn't have}} a headache \textit{\small{-Forma Américana}}
                \item [-]You \textcolor{red}{\textbf{hasn't got}} a headache \textit{\small{-Forma Británica}}
            \end{itemize}
        \end{example}
        \begin{note}
            \\En \textbf{Americano}:
            I don't have a car
            \\En \textbf{Británico} (más formal):
            I haven't got a car.
            \\Ojo con el uso del \textbf{does} o \textbf{doesn't} el verbo cambia, ejemplo el has pasa a have.
            \\La contracción solo se puede usar si va con el \textbf{got}, si no lo lleva entonces toca escribir el have completo
            \\\textcolor{green}{I've got blue eyes} (bien) \textcolor{red}{I've blue eyes}(mal)
        \end{note}
        \section{Adverbs Of Frecuency}
        \begin{corollary}{Syntax}{}
            \dSmiley[2] + adverb + main verb\\
            \dSmiley[2] + Aux + adverb + main verb\\
            \dSmiley[2] + BE  + adverb
        \end{corollary}
        \textcolor{red}{\textbf{Pegar imagen de adverbios de frecuencia}}
        \begin{property}
            Normalmente va antes del verbo a menos que exista TO-BE
        \end{property}
        \begin{property}
            Los siguientes pueden ir al comiezo, antes del \dSmiley[2]:
            \begin{itemize}
                \item Usually
                \item Normally
                \item Frequently
                \item Generally
                \item Occasionally
                \item Sometimes
            \end{itemize}
        \end{property}
        \begin{property}
            Los siguientes \textbf{\textcolor{red}{NO}} pueden ir al comiezo, antes del \dSmiley[2].
            \begin{itemize}
                \item Always
                \item Hardly ever
                \item Seldom
                \item Rarely
                \item Never
            \end{itemize}
        \end{property}
        \begin{example}
            \begin{itemize}
                \item I always work on monday
                \item She never goes to the gym
                \item I have always done my work
                \item I am normally busy at work
                \item He is always happy
                \item Occasionally I have pancakes for breakfast
                \item Normally our English class is in the morning
                \item I always get up early
                \item Do you always have breakfast in your house?
                \item How often do you have breakfast in your house?
                \item Does he often drink milk?
                \item How often does he drink milk?
                \item Do they frequently go to the gym?
                \item How frequently do they go to the gym?
                \item Do they normally work on weekends?
                \item How often do they work on weekends?
                \item Do you study English very often?
                \item How often do you study English?
            \end{itemize}
        \end{example}
        \section{Time Expressions}
        \begin{property}
            Van al inicio o final de una frase, normalmente al final.
        \end{property}
        \begin{property}
            \emph{Once} y \emph{Twice} van sin el \emph{times}.
        \end{property}
        \begin{itemize}
            \item Every day/month/year
            \item Once/Twice a day/week/month/year 
            \item Three \textcolor{red}{\textbf{times}} a day/week/month/year
            \item Four  \textcolor{red}{\textbf{times}} a day/week/month/year
            \item In the Morning/Evening/Afternoon
            \item At night
            \item On monday
            \item In July
        \end{itemize}
        \section{There is - There are}
        \begin{property}
            There is -> Hay, existencia, Singular
        \end{property}
        \begin{property}
            There are -> Mismo significado pero en Plural
        \end{property}
        \begin{property}
            Any normalmente para plural o negativo
        \end{property}
        \vspace{1em}
        \begin{tabular}{|ccc|}
            \hline
            \textbf{Positive}&\textbf{Negative}&\textbf{Interrogative}\\
            \hline
            There is a table&Isn't a table&Is there a table?\\
            There are two tables&Aren't \underline{any} table&Are there \emph{any} desk?\\
            \hline
        \end{tabular}
        \section{Prepositions}
        \begin{property}
            Normalmente no es ncesario poner el "of", lo tienen incluido, solo algunos si lo necesitan.
        \end{property}
        \begin{itemize}
            \item Under
            \item Behind
            \item Above
            \item In front of
            \item Next to
            \item Between
            \item In
            \item On
        \end{itemize}
        \subsection{Prepositions in time \emph{IN, ON, AT}}
        \subsubsection*{In}
        \begin{property}
            Contención ó adentro
        \end{property}
        \begin{property}
            Use \emph{in} with \textbf{seasons} as a \emph{winter}
        \end{property}
        \begin{property}
            Use \emph{in} with \textbf{months}
        \end{property}
        \subsubsection*{On}
        \begin{property}
            Superfice ó encima
        \end{property}
        \begin{property}
            Use \emph{on} with special days, \emph{on easter monday}
        \end{property}
        \begin{property}
            Use \emph{on} when you write the fulldate, \emph{on 4th July, 1998}
        \end{property}
        \subsubsection*{At}
        \begin{property}
            Ubicación o grado de medida
        \end{property}
        \begin{property}
            Use with the time of the day, \emph{at 4pm}
        \end{property}
        \begin{property}
            Use with festivals, \emph{at Christmas}
        \end{property}

        \begin{example}
            \begin{itemize}
                \item Water boils \textbf{at} $100^0C$
                \item Wait for me \textbf{at} 4pm
                \item I'm \textbf{at} the corner
                \item I have coins \textbf{in} my pocket
                \item Lucy is \textbf{in} Chicago
                \item I left my keys \textbf{on} the table
                \item She left a note \textbf{on} the desk
            \end{itemize}
        \end{example}


        \section{Can \emph{for possibility}}
        \begin{property}
            Si uso el \textcolor{red}{\textbf{CAN}} entonces el verbo que esta al lado va normal, sin conjugarlo, en infinitivo, en forma base.
        \end{property}
        \newpage
        \section{A, An, Some, Any}
        \begin{property}
            \emph{a} and \emph{an} use with \textbf{singular, countable nouns}
        \end{property}
        \begin{property}
            \emph{Some} and \emph{Any} for \textbf{countable and uncountable}
        \end{property}
        \begin{property}
            Some
            \begin{itemize}
                \item Para ofrecer cosas ó pedir cosas
                \item Would you like \emph{some} water?
                \item Can I have \emph{some} sugar?
            \end{itemize}
        \end{property}
        \begin{property}
            Any
            \begin{itemize}
               \item Se usa para preguntas en General 
               \item Are there \emph{any} students in the class? 
            \end{itemize}
        \end{property}
        \begin{example}
            \begin{itemize}
                \item We need \textbf{some} apples
                \item There isn't \textbf{any} juice
                \item The coffee needs \textbf{some} sugar
                \item There aren't \textbf{any} students in the class
                \item The pot doesn't have \textbf{any} salt
                \item Do you want \textbf{some} apples?
                \item Is there \textbf{any} of water in the bottle?
                \item Woud you like \textbf{some} of milk?
                \item Are there \textbf{any} animals in your house?
            \end{itemize}
        \end{example}
        \section{NO OLVIDAR AGREGAR UNA PAGINA EXCLUSIVA PARA VOCABULARIO Y OTRA PARA PRONUNCIACIÓN}
        \section{Quantifiers}
        \begin{corollary}{}{}
            How much + uncountable nouns\\
            How many + plural countable nouns\\
            A lot / lots (of) / quite a lot (of) / not much/many + noun
        \end{corollary}
        \begin{property}
            \emph{How much} - Always is for singular
        \end{property}
        \begin{property}
            \emph{How many} - Always is for plural
        \end{property}
        \vspace{0.5em}
        \begin{tabular}{|l|}
            \hline
            A lot of + noun\\
            Lots (of) +  noun\\
            Quite a lot (of)\\
            Not much +  noun\\
            Not many + noun\\
            None\\
            \hline
        \end{tabular}
        \newpage
        \begin{example}
            \begin{itemize}
                \item How much sugar have we got?
                \item How much milk is there in the fridge?
                \item How many tomatoes are there in the bag?
                \item How many vegetables do you eat in a week?
                \item I eat \textbf{a lot of }fruit
                \item I don't drink \textbf{much} water
                \item How much exercise do you do in a week?
                \item How many glasses of water do you drink in a day?
            \end{itemize}
        \end{example}
        \begin{note}
            Use quantifiers for short answers: :\\
            How much cheese have we got? \textbf{none}
        \end{note}





        



\end{document}%fin del documento
